% Instructions to modify this document:
% * Remember to ALWAYS execute a git pull BEFORE any commit you make
% * Use the \ToDo{...} command to remark tasks which still need to be done
% * Use the \input{file.tex} command to split the document into several parts
% * Do not change the current LaTeX coding style to yours

% To convert .dia diagrams into PDF:
% 1) Create the diagram with dia (or any other tool)
% 2) Export it as .eps
% 3) use epstopdf to convert to PDF


\documentclass[a4paper,12pt]{article}

\usepackage[utf8]{inputenc}
\usepackage{amsmath,graphicx}
\usepackage{bm}
\usepackage{amssymb}
\usepackage{algorithm}
\usepackage{algpseudocode}
\usepackage{subfigure}
\usepackage{ifpdf}
\usepackage{url}
\usepackage{color}
\usepackage[hidelinks]{hyperref}
\usepackage{multirow}
\usepackage{datetime}
\usepackage{comment}
\usepackage{float} % To put figures in their exact place with \begin{figure}[H]
\usepackage{longtable}
\usepackage{tabularx}

\newcolumntype{L}[1]{>{\raggedright\arraybackslash}p{#1}}
\newcolumntype{C}[1]{>{\centering\arraybackslash}p{#1}}
\newcolumntype{R}[1]{>{\raggedleft\arraybackslash}p{#1}}


% Definitions and commands
\def \np{\vskip 0.25 cm}
\def \ap{\vskip 0.15 cm}

\newcommand{\ToDo}[1]{\textcolor{magenta}{\textbf{[ToDo]} \textbf{#1}}}


\begin{document}


\begin{titlepage}

\begin{center}
\vspace*{-1in}

%Universitat Oberta de Catalunya\\
\vspace*{0.6in}
\begin{Large}
\textbf{The IPOL Demo System 2.0 \\Technical documentation} \\
\end{Large}

\vspace*{0.6in}

\small{Compiled on \today\ at \currenttime}

\vspace*{0.6in}
\rule{80mm}{0.1mm}\\
\vspace*{0.1in}
\end{center}

\end{titlepage}

This documents contains technical documentation for the architecture and modules of the IPOL Demo System 2.0, made by the architecture and software engineering team.
\vspace*{0.6in}

\textbf{Project direction and team coordination}

Miguel Colom - \url{http://mcolom.info}

\vspace*{0.2in}

\textbf{Software engineers (alphabetic order)}

José Arrecio

Miguel Colom

Vincent Firmin

Karl Krissian

Alexis Mongin

Nelson Monzón


%\maketitle
\newpage

\tableofcontents
\newpage
\listoffigures
\newpage

\section{Introduction}
\ToDo{Introduction}

\section{The Demo System Core}
Centralized webservice. Dispatcher. Proxy pattern \cite{GoF}.

The list of datatypes used (images, audio, video, 3D pointclouds, 3D meshes) and how to identify them. The "no type" format for demos which have their own non-standard format.

\ToDo{Document it!}

% The Blobs module
\section{The Blobs module}

\subsection{Introduction}
\label{sec:blobs_introduction}
The Blobs module introduction


% The Archive module
\section{The Archive module}

\subsection{Introduction}
\label{sec:archive_introduction}

%\paragraph{Introduction} \hspace{0pt} \\
The archive module is a standalone application destined to communicate with other modules using webservices. It is designed to implement a stable, simple and scalable system for archiving all experiments done with IPOL.

\paragraph{Technologies used} \hspace{0pt} \\
The archive module, written in Python, is using the cherrypy framework for webservices, the mako template library for webpage rendering, Python Image Library for thumbnails creations, and the python-magic library available on pip (not to be mistaken with python-magic5 which is the one available on ubuntu's default APT repositories). The module communicates using JSON, both in input and output. The database engine used is SQLite.

\subsection{Architecture}

\paragraph{Module composition} \hspace{0pt} \\
The module is composed of very few files, the code itself in ``module.py'', a cherrypy configuration file ``archive.conf'', two mako HTML templates, and a database. It will also need 4 directories, respectively for storing blobs, thumbnails, the database and logs.

\paragraph{Module architecture} \hspace{0pt} \\
The module is composed of a class, 'Archive', encapsulating the data needed to function. The services offered by the module are all methods of this class. The cherrypy framework provide the abstraction for making the methods available as webservices. \\
Upon starting the module, the cherrypy engine is launched, an object of the Archive class is created, and the cherrypy configuration is loaded from ``archive.conf''. If they don't exist, both the database and the directories needed for the storage of blobs, logs, the database and thumbnails will be created, provided that the user launching the module has the necessary rights. Otherwise, the module will not start. These directories are indicated in the cherrypy configuration for maximum configurability, if they are missing from it, the module will not start. \\
The webservices communicate with the server via arguments given through URL, as unicode strings directly passed to the methods. \\
The services all connect to the database in a thread-safe way, instanciating its own connection when called, commiting when done if there are modifications, or rollbacking, if there is an error, and closing the connection. \\
There is a logger initialised with the Archive object, writing errors in ``error.log'' in the logs directory given in the configuration file.

\subsection{Database design}

The database contains 3 tables : experiments, blobs, and correspondence.\\
Each experiments, and each blobs are defined individually, and linked to each-others in the correspondence table, assuring a many-to-many connection. It is worth noting that the database does not save duplicates of the same blob. \\

\ToDo{[Miguel] The right name for the ``correspondence" table is \emph{Junction table}. You can keep the name ``correspondence", but explain in the text that it is a junction table.}

\begin{tabular}{|l|c|r|}
  \hline
  experiments & blobs & correspondence \\
  \hline
  id & id & id \\
  id\_demo & hash & id\_experiment \\
  params & type & id\_blob \\
  timestamp & format & name \\
  \hline
\end{tabular} \\

\paragraph{Experiments table} \hspace{0pt} \\
The experiments table is defined as such : the id field, that stores the unique id of the experiment ; the id\_demo field, that stores the id of the IPOL demo used for the experiment ; the params field, which is a JSON string whose format varies from demo to demo ; and finally the timestamp field.

\paragraph{Blobs table} \hspace{0pt} \\
The blobs table is defined as such : the id field, that stores the unique id of the blob ; the hash field, that stores the hash of the blob computed with sha1, the type field, that stores the extension of the blob (e.g. ``jpeg'' or ``png''), and the format field, that stores the media format of the blob : it is a string, either ``audio'', ``video'' or ``image''. \\
The physical location of a blob is ``blob\_dir as defined in the configuration file'' + ``hash of the blob'' + ``.'' + ``type of the blob''.

\paragraph{Correspondence table} \hspace{0pt} \\
\ToDo{[Miguel] You can keep the name ``correspondence", but explain in the text that it is a junction table.}
The correspondence table is defined as such : the id field ; the id of the experiment and the id of the blob that is linked to said experiment, and the name field, which indicates the role of the blob in the experiment (example : ``input'' or ``denoised''). A foreign key constraint allowing cascade delete is put on the field id\_experiment, referencing the id of an entry in the experiment table, for automatic data deletion.

\subsection{Services}

\paragraph{Adding an experiment to the archive} \hspace{0pt} \\
Example :
\begin{verbatim}
http://<localhost>:<port>/add_exp_test?demo_id=42&blobs=
<json_blobs>&parameters=<json_parameters>
\end{verbatim}
The method ``add\_experiment'' takes in the entry of the id of the demo used ; a JSON string of the format : 

\begin{verbatim}
{
    url_blob : name,
    ...
}
\end{verbatim}

containing a description of each blob used by and produced by the experiment, with their temporary URLs and names ; and a JSON string describing the parameters of the demo used for the experiment. It will add an experiment to the database by creating a new entry in the experiment table. If the blobs used by and produced by the experiment aren't already in the database, it will copy them in the directory given in the configuration file, and create a thumbnail for the images. It will return a json string containing the status of the operation, OK if it succeeded, KO if there was an error and the operation wasn't performed, as such :

\begin{verbatim}
{
    status : OK/KO
}
\end{verbatim}

If status is KO, a log describing the error will be written.

\paragraph{Deleting an experiment from the archive} \hspace{0pt} \\
When removing an experiment from the database via the method ``delete\_experiment'', every blob linked to this experiment and only to this experiment is removed. After that, all the entries in the correspondence table referencing this experiment are removed automatically due to a foreign key constraint. It return a json response containing the status of the operation of the same format as the return of the method ``add\_experiment''. The method shouldn't be called anywhere else than through the user interface described later.

\paragraph{Deleting a blob from the archive} \hspace{0pt} \\
Due to a many-to-many link between blobs and experiments in the database, a blob has a lot of dependencies : it has of course the experiments using this blobs, but also the blobs linked to these experiments. For deletion of a blob from the archive, the precedent service is called on each experiment the blob is part of, assuring that no orphan data stay in the database (e.g. experiments linked to removed blobs or blobs linked to removed experiments). The method implementing this service is ``delete\_blob\_w\_deps''. It return a json response containing the status of the operation in the same format as the return of the method ``add\_experiment''. The method shouldn't be called anywhere else than through the user interface described later.

\paragraph{Getting data \ToDo{[Miguel] use a more specific word than ``data"} from an archive page} \hspace{0pt} \\
Example :
\begin{verbatim}
http://<localhost>:<port>/page?demo_id=42&page=3
\end{verbatim}
The method ``page'' returns a JSON response with, for a given page of a given demo, all the data of the experiments that should be displayed on this page. Twelve experiments are displayed by page. For rendering the archive page in the browser, the JSON response should be parsed and interpreted in a dedicated template furnished by the front-end of another module. The JSON response is formatted this way : 
\begin{verbatim}
{
    status :  OK/KO,
    experiments : [
        {
            date : timestamp_example, 
            files : [
                {
                    url : url_example,
                    id : id_example,
                    name : name_example,
                    url_thumb : url_thumbnail_example
                }
            ... ],
            id : id_example,
            parameters = {parameters_example...}
    ... ],
    id_demo : id_demo_example,
    nb_pages : nb_pages_example
}
\end{verbatim} 

\paragraph{Administrator interface for removing blobs/experiments} \hspace{0pt} \\
Example :
\begin{verbatim}
http://<localhost>:<port>/archive_admin?demo_id=42&page=3
\end{verbatim}
The only user interface furnished by the archive module is for removing blobs or experiment in a convenient manner. It uses the json response of the precedent service and renders the ``archive\_admin\_tmp.html'' template displaying a page of archives for given demo, allowing the deletion of both blobs and experiments by simply linking to two other services calling deletion methods and updating the template. In case of error, for example when invalid data is given through URL, ``error.html'' is rendered.

\paragraph{Shutdown} \hspace{0pt} \\
Example :
\begin{verbatim}
http://<localhost>:<port>/shutdown
\end{verbatim}
The method ``Shutdown'' shuts down the archive application when called. It returns a json response containing the status of the operation.

\paragraph{Other services} \hspace{0pt} \\
Other services features the method ``ping'', simply for checking if the module is up, and the method ``stats'', formatted this way :
\begin{verbatim}
{
    status : OK/KO,
    nb_experiments : x,
    nb_blobs : y
}
\end{verbatim}
Example :
\begin{verbatim}
http://<localhost>:<port>/ping
\end{verbatim}
Example :
\begin{verbatim}
http://<localhost>:<port>/stats
\end{verbatim}


% The Demo Description Language and automatic demo generation
\section{The Demo Description Language}

current version is 1.0

\subsection{Introduction}
The Demo Description Language (DDL) is a textual description of a IPOL demo that 
allows to compile, generate a web interface, and run a demo. The description is 
written in JSON (JavaScript Object Notation) format, which is a standard format 
used in web applications. This language is evolving to allow a maximum number 
of demos to be described without the need to manually write HTML or Python code 
for a given demo. Each main key in the description file is described in the 
following sections.


{\bf Note:} in JSON format, always use double quotes around keys or string 
values, not single quotes.

%-------------------------------------------------------------------------------
\subsection{The \emph{general} section}
The general section describes global information about the demo. It is a set of 
(key,value) pairs, described in the following table. Many keys are derived from 
the static variables of the previous 'app' Python class.

\begin{longtable}{|>{\bf}L{\dimexpr 0.27\linewidth}|L{\dimexpr 
0.58\linewidth}|c|}
\hline
 \centering {key}     & \centering {\bf description} & {\bf req} 
\tabularnewline \hline \hline
 demo\_title         & demo title & yes\\ \hline
 input\_description  & description at the top of the input selection page 
                     & yes \\ \hline
 param\_description  & description at the top of the parameters page & yes
                      \\ \hline
 input\_nb           & number of inputs (currently only 1 is supported) & yes\\ 
\hline
 input\_max\_pixels & sets the maximal number of pixels of the input image, 
bigger images will be downsized, if 0 no resizing is done. The resizing as 
defined in the image class uses python PIL resizing with 'antialias' option. & 
yes  \\ \hline
 input\_dtype       & input image expected data type, used as parameter of 
image.convert() method. Possible values are '1x8i' and '3x8i' which are 
converted respectively to 'L' and 'RGB' for PIL.& yes  \\ \hline
 input\_max\_weight & max size (in bytes) of an input file, prevents uploading 
bigger files. & yes  \\ \hline
 input\_ext         & input image expected extention (ie. file format) & yes  
\\ \hline
 is\_test           & boolean (true/false), if true will be removed from demos 
if the server is in production mode.& yes  \\ \hline
 xlink\_article     & defines the link to the article webpage & yes  \\ \hline
\caption{Keys for the 'general' section ({\em req} means required).}
\end{longtable}

%-------------------------------------------------------------------------------
\subsection{The \emph{params} section}
The params section describes the set of parameters needed by a demo, their 
constraints and their visual appearance. It is defined as an array of sets, 
where each set contains (key,value) pairs.

\subsubsection{Common values}

\begin{longtable}{|>{\bf}L{\dimexpr 0.27\linewidth}|L{\dimexpr 
0.58\linewidth}|c|}
\hline
 \centering {key}     & \centering {\bf description} & {\bf req} 
\tabularnewline \hline \hline
 type  & parameter type, one of \{ label, range, selection\_collapsed \}.
                     & yes \\ \hline
 label & description of the parameter or contents if type is label. & yes
                      \\ \hline
\caption{Common keys for the 'params' section ({\em req} means required).}
\end{longtable}


\subsubsection{ \emph{label} type}
The label type does not need any other value. It can be used as a title to 
separate groups of parameters.

\subsubsection{ \emph{selection\_collapsed} type}


\begin{longtable}{|>{\bf}L{\dimexpr 0.27\linewidth}|L{\dimexpr 
0.58\linewidth}|c|}
\hline
 \centering {key}     & \centering {\bf description} & {\bf req} 
\tabularnewline \hline \hline
 id     & parameter name  & yes \\ \hline
 values & description of the parameter or contents if type is label. & yes
                      \\ \hline
 default\_value &  & yes \\ \hline
 value & & yes \\ \hline
\caption{Additional keys for the 'selection\_collapsed' type.}
\end{longtable}

\subsubsection{ \emph{range} type}

\begin{longtable}{|>{\bf}L{\dimexpr 0.27\linewidth}|L{\dimexpr 
0.58\linewidth}|c|}
\hline
 \centering {key}     & \centering {\bf description} & {\bf req} 
\tabularnewline \hline \hline
 id     & parameter name  & yes \\ \hline
 values & description of the parameter or contents if type is label. & yes
                      \\ \hline
 default\_value & & yes \\ \hline
 value & & yes \\ \hline
\caption{Additional keys for the 'range' type.}
\end{longtable}

%-------------------------------------------------------------------------------
\subsection{The \emph{results} section}

%-------------------------------------------------------------------------------
\subsubsection{ \emph{run\_again} type}

\subsubsection{ \emph{warning} type}
\subsubsection{ \emph{image} type}
\subsubsection{ \emph{repeat\_image} type}
\subsubsection{ \emph{gallery} type}
\subsubsection{ \emph{repeat\_gallery} type}
\subsubsection{ \emph{html\_text} type}
\subsubsection{ \emph{file\_download} type}

%-------------------------------------------------------------------------------
\subsection{The \emph{build} section}
\subsubsection{url}
The full url link to download the demo source code.
\subsubsection{srcdir}
Subdirectory from the extracted archive where the source code is located.
\subsubsection{binaries}
List of binaries and their associated paths relative to the source code path.
\subsubsection{flags}
Cmake compilation flags.
\subsubsection{scripts}
List of scripts and their associated paths relative to the source code path.

\subsection{The \emph{run} section}
\subsubsection{keywords}

\subsection{Examples}


\subsubsection{keywords}


\section{DemoInfo module}
This module store the textual description of the demo and allows to ask for specific sections of it. It also stores other demo-related information, as the demo authors, their emails, etc.

It can obtain the system load from {\tt /proc/loadavg}. The first three numbers are the load averages for the past 1, 5, and 15 minutes. Load averages are not normalized for the number of CPUs in a system, so a load  average  of 1 means a single CPU system is loaded all the time while on a 4 CPU system it means it was idle 75\% of the time.
\ToDo{Document it!}

\section{DemoDispatcher module}
In order to distribute the load in several machines, this module checks the work load of each known DemoRunner modules and starts an algorithm demo execution on the less loaded machine. The process is done transparently and from the outside this is the only visible module, since the actual DemoRunners are not directly accesible.
\ToDo{Document it!}

\section{DemoRunner module}
The DemoRunner module has two main tasks. The first is to inform the DemoDispatcher about the load of the machine where it is running. The second task is to execute an algorithm given the parameters and store the results in a temporary directory. It is a requirement that the temporal results can be accessed by different users using the URL (for example, it may contain a key which identifies a specific experiment).
\ToDo{Document it!}

% The Control Terminal 
\subsection{The Control Terminal}

The Control Terminal is an standalone application intended for system administration which allows to start, stop, and query the status of each of the IPOL modules. It reads the IPOL configuration from the XML files at {\tt ipolDevel/ipol\_demo/modules/config\_common}.

\subsubsection{Structure}
The Control Terminal is a command-line application to control the IPOL modules at each environment. The terminal is set to particular  enviroment to send the commands to the server in that environment. For example, the local, integration, or production environments. The current environment can be get or set with the {\tt env} command.

\paragraph{XML files} \hspace{0pt} \\
The XML files {\tt modules.xml} and {\tt demorunners.xml} are used by the IPOL modules when started. The Terminal reads these configuration files when started or when the environment is changed.

In {\tt modules.xml} the fields are:
\begin{itemize}
    \item module: the name of the module
    \item server: the host name of the server
    \item serverSSH: the name of the server (used to ssh it)
    \item path: the physical path of the module in the server
    \item command: it declares a command that the module is able to execute
\end{itemize}

The {\tt demorunners.xml} lists and configures each of the demoRunners in that enviroment.

\subsubsection{Commands}
\paragraph{start} \hspace{0pt} \\
Usage: {\tt start <module>}

It starts the specified module by ssh'ing to the server where the module is physically located and invoking the {\tt start.sh} script.
The {\tt ping} command might be executed right after to check if the module is indeed up.

\paragraph{ping} \hspace{0pt} \\
Usage: {\tt ping <module>}

It pings the module to check that it is responsive.

\paragraph{shutdown} \hspace{0pt} \\
Usage: {\tt shutdown <module>}

It shutdowns the specified module.

\paragraph{info} \hspace{0pt} \\
Usage: {\tt info <module>}

It prints the list of available commands for the specified module.

\paragraph{modules} \hspace{0pt} \\
Usage: {\tt modules}

It displays the list of the modules in the IPOL system.

\paragraph{env} \hspace{0pt} \\
Usage: {\tt env <environment>}

It prints the current enviroment when called without any parameters, or sets the specified enviroment.

\paragraph{help} \hspace{0pt} \\
Usage: {\tt help}

It prints the help of the Terminal.


\section{The Control Panel web application}
In the modular architecture of IPOL all the modules are able to work standalone since they are RESTful servers implementing webservices. However, a coordinated view from both the functional and administrative point of view is needed.

The Control Panel web application offers a unified interface to configure and administrate the platform. For example, the Editors can remove undue experiments for the archive or add new images to a demo.

\ToDo{Document it!}

\section{Automatic testing}
Two kind of tests:
\begin{itemize}
  \item For the IPOL system itself
  \item To ensure good compilation of the algorithm source codes
\end{itemize}

\section{Implementation details}
This section discusses some implementation details and specific configurations needed.

\begin{itemize}
  \item The magic module. We use PIP, not the python-magic package found in most distributions. \ToDo{Explain why}.
  \item List of packages needed and how to install them.
  \item Description of the general requirements needed for a module to be part of the system (start, ping, and shutdown services ; structure of the general launching scrip...
\end{itemize}

\ToDo{Keep adding!}


\bibliographystyle{plain}
\bibliography{biblio}

\end{document}
% End of document

