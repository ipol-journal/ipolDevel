% Project management and development methodology

\section{Project management}
\label{sec:methodology}
The current IPOL project tries to follow the best practices in software engineering. Specifically, for this kind of project we found that Continuous Integration was a good choice in order to achieve fast delivery of results and ensuring quality. Continuous Integration is a methodology for software development proposed by Martin Fowler~\cite{fowler2006continuous} which consists on making automatic integrations of each increment achieved in a project as often as possible in order to detect failures as soon as possible. This integration includes the compilation and software testing of the entire project.

It is a set of policies that, together with continuous deployment, ensures that the code can be put to work quickly. It involves automatic testing in both integration and production environments. In this sense, each contribution in the IPOL system is quickly submitted and several automatic test are performed. If any of these tests fail the system sends an email indicating the causes.
%
Another advice of Continuous Integration is minimal branching. We use two. On one hand, master is the default branch and where all the contributions are committed. It is used for the development, testing and this continuous integration; on the other hand, the prod branch is used only in the production servers. It is merged with master regularly.
%
We use two different environments: integration and production. The integration server is where the master branch is pulled after each commit. The prod branch is used for the production servers and the code in this branch is assumed to be stable. However, the code in the integration server is also assumed to be stable and theoretically the code in the master branch could be promoted to production at any time once it has been deployed to the integration server and checked that is is fully functional and without errors. 

Quality is perhaps the most important requirement in the software guidelines of the IPOL development team. The code of the modules must be readable and the use of reusable solution is advised~\cite{GoF}. The modules must be simple, well tested and documented, with loose interface coupling, and with proper error logging. Note that it is not possible to ensure that any part of the IPOL will not fail, but in case of a failure we need to limit the propagation of the problem through the system and to end up with diagnostic information which allows to determine the causes afterwards.
%
Refactoring~\cite{fowler1999refactoring} is performed regularly and documentation is as important as the source code. In fact, any discrepancy between the source code and the documentation is considered as a bug.

Another tool used by the team is Trello. It allows to track the tasks of the project accoding to their current state (not assigned, assigned but not stated, assigned and in development, and finished). When a task arrives to the ``Finished" step, it is reviewed by the Project Coordinator and archived (task totally finished) or moved back to development if more work is needed.

\subsection{Development methodology}
In the git repository there are two different branches:
\begin{itemize}
	\item \textbf{devel}, to develop new features or any changes; and
	\item \textbf{master}, where contribution are integrated.
\end{itemize}

The devel branch must be always in a \emph{green} state, meaning that any contributions added to devel might be added immediately to master. The Integration server has devel checkout out, and the production servers have master.

The way to contribute to the project is as follows:
\begin{enumerate}
	\item Create a feature branch (FB) from devel;
	\item Commit your contributions to the FB;
	\item When ready, make a pull request (PR) from your FB to the devel branch;
	\item When your contribution has been validated, you can merge the FB into devel;
	\item Make the Integration server pull the last changes with the IPOL's terminal tool;
	\item Make any changes in the Integration server if needed (changing nginx configs, updating virtual environments, or others);
	\item Test carefully in the Integration server that everything works correctly, especially your contribution;
	\item The coordinator will merge from master to devel, pull the commits in master into the production servers with the IPOL's terminal in one of the regular meetings, and all the team will check that everything works as expected in production.
\end{enumerate}

For minor contributions, you can proceed directly without waiting for the approval of the coordinator.
For contributions which are major, you need to discuss them first with the team internally, and wait until an official meeting with all the team is made to decide if it is accepted. The coordinator will have the last word.

The validation by the peers should be complete. The team must check that there is not a solution which might be more suitable, if refactoring is neeed, that the documentation has been modified accordingly, or that the configurations have been updated, to cite some examples.
