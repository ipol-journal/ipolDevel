% Nginx as a reverse proxy

\section{Nginx as a reverse proxy}
\label{sec:reverse_proxy}
The IPOL project uses Nginx as a reverse proxy in order to redirect internet requests to the servers in the 
internal network. When Nginx receives a request, it decides which module it hast to send and so which machine that module is, 
then fetches the response, and sends it back to the client. With Nginx we implement private demos, microservise architecture 
patter and serve static files used by the clients.

\subsection{Static files}
The control panel as a Django web application needs to be compiled everytime the files change, so in order to serve the 
application Django offers a cli tool to collect all static files. Nginx serves all the static files used by the application and exposes them 
under a public route.

\subsection{API routing}
Nginx is also used to redirect incoming requests to the corresponding port and direction inside the machine where the module is 
running. Each modules has an internal file (sites-available) that decides where to route the request and deliver it to the desired endpoint.

\subsection{Private demos}
The IPOL system provides private demos that requires authentication by username and password. The ID of these demos begin with 
33333001. In this sense, Nginx can detect when a private demo is requested and it can forbid the access if the authentication fails.