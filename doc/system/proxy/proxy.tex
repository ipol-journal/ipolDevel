% Nginx as a reverse proxy

\section{Nginx as a Reverse Proxy}
\label{sec:reverse_proxy}

The demo system is a distributed architecture of isolated units which communicate with an HTTP-based API. Thanks to this API these
units do not need to know which is the location in the network of the rest of other units. As explained in Sec. \ref{sec:methodology} this
does not only hide the complexity of the system from the outside, but from the inside too.

To implement the requests routing with API to the corresponding modules, a \emph{reverse proxy} is used. A reverse proxy receives a
request, analyses it, and forwards it to the designated server using a table. It is the inverse operation which is performed by a
classic proxy. In the IPOL demo system the nginx reverse proxy is used.

Nginx is also used to redirect incoming requests to the corresponding port and address of the machine where the module is 
running on. Each modules has an internal file (sites-available) that decides where to route the request and deliver it to the desired endpoint.

With Nginx the system implements private demos, the API routing, and serve static files. The IPOL system provides private demos which require
authentication using a username and password. The system decides if a demo is private or public depending on its ID. To implement this
mechanism, Nginx checks the argument ID from the url looking for a numeric pattern, analizes this number, redirects to the authorization
page and depending on the company name it will contrast the information provided by the user against this company credentials. If it
is successful the system will grant access, otherwise an error will be shown. The following configuration example checks the ID of the
demo and redirects the authentication process if the pattern matches. 

\begin{lstlisting}[language=Bash]
location /demo/clientApp/ {
        expires 24h;
        if ($args ~ ".*id=33333001\d*") {
            error_page 418 = @CompanyNamePasswd;
            return 418;
        }
        alias  /resource_folder/;
    }
 \end{lstlisting}


Nginx serves static files and exposes them under a public route. The control panel is a Django web application and it needs to be compiled 
everytime the files change. In order for the web browsers to reach that code it has to be publicly available. The system serves these
files using Nginx static files declarations. Apart from the control panel files, Nginx also serves static files for the demo system, this includes
blobs and archive modules.