
\section{DemoRunner module}
\label{sec:DemoRunner}

This module controls the execution of the IPOL demos. The DemoRunner module is responsible of informing the Core about the load of the machine where it is running, of ensuring that the demo execution is done with the last source codes provided by the authors (it downloads and compiles these codes to maintain them updated), and of executing the algorithm with the parameters set by the users. It takes care of stopping the demo execution if a timeout is reached, and to inform the Core about the causes of a demo execution failure so the Core can take the best action in response.

The DemoRunner module has two main tasks. The first is to inform the Dispatcher about the load of the machine where it is running. The second task is to execute an algorithm given the parameters and store the results in a temporary directory. It is a requirement that the temporal results can be accessed by different users using the URL (for example, it may contain a key which identifies a specific experiment).

It can obtain the system load from {\tt /proc/loadavg}. The first three numbers are the load averages for the past 1, 5, and 15 minutes. Load averages are not normalized for the number of CPUs in a system, so a load  average  of 1 means a single CPU system is loaded all the time while on a 4 CPU system it means it was idle 75\% of the time.
\ToDo{Document it!}

