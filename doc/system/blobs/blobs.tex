\section{The Blobs module}

Note : For it is easy to get confused about such matter, we refer as modules of the IPOL system as ``modules'' and python module (files containing python code) as ``python module''

\subsection{Introduction}
Each demos of IPOL can be presented to the user with a set of defaults blobs. Thus, the user isn't forced to supply their own files for seeing the executions of the algorithms in the browser. These defaults blobs can be tagged and linked to differents demos. That way, blobs with interesting features (like blur or noise) can be grouped together, and the user will be free to see the differences between multiple algorithms applications on the same blobs. All of this mecanism is implemented by the blobs module, making it a critical part of the system.

\subsection{Composition}
The blobs module is composed of several pythons modules implementing different functionalities. It is composed of the same technical stack as the rest of the back-end architecture of the IPOL webapp : written in Python2.7, using the cherrypy web framework, and SQLite as database engine, and using the Mako templates for generating HTML responses for webservices destinated to humans.
\ToDo{[Alexis]: information about the technical stack is redundant with the archive section, since this is the same stack. Maybe we should move that in a more general section.}

\subsubsection{Main python module}
The main module is called by ``start.sh'', the script called by the control terminal to get modules running on different servers over ssh. It consist of setting up some variables used by cherrypy, as well as mounting the blobs class at the root of the used server.

\subsubsection{Error python module}
This module describe the errors used by the Blobs module and add color to the error messages printed in the terminal. Since the module implement logging, it is very likely to be outdated and should be refactored soon.

\subsubsection{Database and database python module}
The first and most important design constraint of the blobs module was that data shouldn't be duplicated where the module is deployed. All blobs referenced by multiple demos should exist in only one directory. This is addressed by establishment of one relational database, referencing the demos, the blobs, and the relations between them. This database also permit to tag some text on blobs, as well as organise them in named sets. In the big picture, the database python module (we also talk about it below) a implement a simple CRUD interface for manipulating the database. Blobs can be managed individually, and upon the deletion of a demo, the blobs related to this demo will be deleted from disk if and only if they were uniquely linked to this demo. Blobs are referenced via their hashes a lot, making easy to verify if a blob is already in the database, even under another filename.\\

The schema of the database is constituted of these tables : \\

demo : 

\begin{tabular}{ | l | l | l | l | }
  \hline
  id & name & is\_template & template\_id \\
  \hline
\end{tabular}\\

blob :

\begin{tabular}{ | l | l | l | l | l | l | }
  \hline
  id & hash & format & extension & title & credit \\
  \hline
\end{tabular}\\

tag :

\begin{tabular}{ | l | l | }
  \hline
  id & name \\
  \hline
\end{tabular}\\

blob\_tag :

\begin{tabular}{ | l | l | }
  \hline
  blob\_id & tag\_id \\
  \hline
\end{tabular}\\
On this junction table, both fields are foreign keys, referencing primary keys in the blob and tag tables.\\

demo\_blob :

\begin{tabular}{ | l | l | l | l | l | }
  \hline
  id & blob\_id & demo\_id & blob\_set & blob\_pos\_in\_set \\
  \hline
\end{tabular}\\
Blob\_id and demo\_id are foreign keys, referencing primary keys in the blob and demo tables. This is a junction table, referencing which blobs and which demos are associated. It allow us to organize the blobs in sets linked to demos.\\

As we can see, the design of the database is fairly simple, with the tables demo, blob and tag storing the information about the system and two junction tables linking this information.

\paragraph{The database python module\\}
This module offer us an interface for accessing the database. One object of the class Database should be instanciated for each operation modifying it. Even if it has function for connecting and closing, they should not be used as such, for flow control issues (for example, an exception leaving a connection open). A very simple abstraction, the DatabaseConnection class is present in the blobs python module for it, and allow safe connection to the database, ensuring they will always be closed no matter what. \\

Otherwise, this python module offer us a wide variety of simple methods for interacting with the database, or obtaining metrics of it, such as the total number of blobs. They generally return the information asked if such case apply. For the format of the responses and the different function, we refer the reader to the code of the module itself. It is worth noting that one webservice, delete\_blob\_from\_demo, recompute the positions of the blobs in a set in which a blob was deleted. If this function is accessed concurrently by multiple threads (the most likely case is if a werbservice calling this function is accessed several times in a very short span), the blobs can end up with miscalculated positions. Non-concurrential access should be enforced by locking the scope where this webservice is called, as it is done in the blobs python module. \\

\subsubsection{Blobs module}

The blobs python module is the core of this. It implement three classes, DatabaseConnection as referenced earlier in the present documentation, and MyFieldStorage, for intermediate storage of the uploaded blobs in the /tmp/ directory, and Blobs as an encapsulation of the webservices and the data they use. It also has some utilitary function. \\

The Blobs class implement all the microservices constituting this module, both those transmitting JSON to other modules, and those generating HTML via mako templates for humans (albeit, the control panel didn't exist when these microservices were implemented and this possible that they will be moved to it soon). An instance of the Blobs class should possess information about the storage of blobs and the networking parameters cherrypy use, such as a port number. A cherrypy configuration file, named ``blobs.conf'', should describe some of these informations. \\

The webservices of the module access the database via instanciations of Database objects managed by the DatabaseConnection class. Some read information, and some modify the database by adding or removing information. For handling a webservice automatically and charging his JSON response as a Python object, the utilitary function use\_web\_service is used several times. \\

Currently, logging is utilised as a mean to retrieve the errors occuring in the system. The logger implemented in the blobs python module handle all the errors of the module. It is passed to each Database object instanciation.\\

Here is a list of all the webservices implemented by the blobs module : \\

\begin{itemize}
\item default : The service invoked when asked for non-existing service.
\item index : web page at the root of where the module is mounted in cherrypy.
\item blob : web page used to upload one blob to one demo.
\item archive : Used to upload one zip file of compressed blobs to one demo.
\item add\_blob\_ws : service checking that the given blobs do not exist in the database. If this is the case, add it.
\item add\_blob : implement the add\_blob page.
\item demos\_ws : return the list of demos from the database.
\item get\_template\_demos\_ws : return the list of template demos from the database
\item demos : web page used to add a demo to the database.
\item set\_template\_ws : webservice used to change the template used by a demo.
\item use\_template : web page used to change the template used by a demo.
\item add\_demo\_ws : web service used to add a demo to the database.
\item add\_demo : web page used to add a demo to the database.
\item add\_from\_archive : webservice used to upload a zip file of compressed blobs to one demo.
\item add\_tag\_to\_blob\_ws : webservice used to add a tag to a blob.
\item op\_add\_tag\_to\_blob : web page used to add a tag to a blob.
\item remove\_tag\_to\_blob\_ws : webservice used to remove a tag from a blob.
\item op\_remove\_tag\_to\_blob : web page used to remove a tag from a blob.
\item op\_remove\_blob\_from\_demo : web page used to remove a blob from a demo.
\item get\_blobs\_from\_template\_ws : webservice used to get the list of blobs from templated demo.
\item get\_blobs\_of\_demo\_by\_name\_ws : webservice returning a list of the hashes of the blobs owned by given demo name.
\item get\_blobs\_of\_demo\_ws : same as the precedent, but with the demo id.
\item get\_blobs\_of\_demo : web page used to display blobs owned by a given demo.
\item edit\_blob : web page showing the thumbnail of a demo with the possibility to add or remove tags.
\item get\_blob\_ws : webservice returning information about a blob from its id.
\item get\_tags\_ws : webservice returning tags of a blob from its id.
\item op\_remove\_demo\_ws : webservice removing a demo from its id.
\item op\_remove\_demo : web page used for removing a demo.
\item ping : used by the terminal for checking module status.
\item shutdown : used by the terminal for turning off the module at distance.
  
  It is worth noting that 2 services unreferenced above, get\_blobs\_of\_demo\_by\_name\_ws, get\_blobs\_of\_demo\_ws, are called by the ``get\_hash'' webservice, which is not in this module.
  
\end{itemize}
