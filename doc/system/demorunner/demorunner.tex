
\section{DemoRunner module}
\label{sec:DemoRunner}

This module controls the execution of the IPOL demos. It can execute directly its \miguel{their} binaries or supporting scripts (provided by the demo editors) related to a particular demo (demoextras). \miguel{The demoExtras have not been explained. For example: the demo can use support code which is not peer-reviewed and belongs only to the demo. We shall refer to this support code as the \emph{demoExtras} and \dots}


Besides, a demo editor can use some generic scripts (PythonTools) that helps for \miguel{help to represent} represent the results of a demo such as draw 2D curves, draw histograms, counting lines and similar \miguel{, among others}. \miguel{What is the difference between demoExtras and those PythonTools? Also, you don't say that they're provided by the system. The explanation is really sketchy.} This \miguel{these} scripts are stored in \miguel{at} each DemoRunner module \miguel{Ok, but it's better to explain that they're accesible at each DR to be called by the demoExtras.}. 

IPOL offers several machines for the demo executions due to its distributed system \miguel{This should be explained in the inverse order: we need several machines in order to execute several demos at the same time by sharing the load among them. This leads to a distributed system with several DRs.}. Thanks to this, the Core can request to run the algorithms into \miguel{at} the machine that best fits according to the \miguel{load balancing policy set in the system.} \miguel{Remove: of the Dispatcher module}. For this, DemoRunner is responsible of informing the Core about the load of the machine where it is running \miguel{running on}. This allows to have several machines with different requirements for the demos (Matlab, specific libraries, etc.) \miguel{Remove: and more concurrent executions due to more machines.}

Once the Core obtains the best machine for the current demo execution, the DemoRunner ensures that the experiment is done with the last source codes provided by the authors. This is achieved by downloading and compiling the source codes \miguel{No, actually the reasons is that it check the dates and the file size, without the need to download adn decompress} (stored in a compressed file) directly from an URL that must be given by the DDL \miguel{Why do you need to do like this? Explain that this ensures that the source code executed by the demo is the same the authors published, thus allowing for Reproducible Research.}.

The first time that a demo is executed, the IPOL system gets the codes from the URL, stores them in the DemoRunner machine and, then, \miguel{remove those commas} it extracts and compiles them \miguel{remove: for the execution}. If this process success \miguel{is successful. 'success' is a noun.}, the module moves the requested executables to the corresponding directory \miguel{'corresponding directory' is quite unspecific.} of the demo and keeps them for the following executions \miguel{Why does it keep them? For the actual execution? To compare dates and sizes? Any other reason?}. Henceforth, IPOL will check modifications in the http headers provided by the URL and the metadata of the codes stored \miguel{It gets the modification date from the HTTP header and compares with its local copy.}. If there are any \miguel{there have been any} modifications \miguel{remove: 'in the content'} or the dates of the file stored in the DemoRunner, IPOL will download and compile the source codes again. \miguel{what about the size of the file?} 

The second intervention of the module is the execution itself \miguel{The second reponsibility of the module is to control the execution. Remove 'the second intervention...'}. The Core provides all the information that the DemoRunner requires\miguel{needs} such as info \miguel{'info' --> information. But simply remove it.} to construct the running path in the shared folder, the DDL run section and the parameters set by the user. \miguel{Also, you refer to the Core. The DR doesn't know about the Core. Write something like: It receives from the execution request all the information it needs to prepare for an execution, such as the the ID of the demo, the execution key, a suggested timeout, and the parameters that the user configured in the web interface.}

Once the execution is \miguel{has} finished, the module responses \miguel{'response' is a noun. Write 'answers'} with the results. DemoRunner provides an interesting \miguel{please keep the documentation technical, without personal observations} mechanism for recovering the info \miguel{'info' is unspecific.}. The editor of a demo can store information from the execution in a text file (algo\_info.txt). The DemoRunner recovers the information \miguel{reads this file and reads the names of the variables and values it contains} and give \miguel{gives it back to the} it to the core\miguel{Core}. Then, the core can return it to the web interface so it can be shown according to the DDL specifications. \miguel{This is not well explained. You need to first explain the problem: the demo needs to send information obtained after the execution back to the web interface. You can put the example of the image gallery in the results, with a variable number of elements. The web interface needs to know how many times it needs to repeat, but this information is only known after the execution. Thus, the demo editor can write this file and add a variable which will be used by the web interface to draw the repeat gallery. (Perhaps you can look for another examples, to have two). Once the problem has been explained, you can explain the solution.}

The module takes care of stopping the demo execution if a problem appears such as not supported inputs for the demo, a bad implementation of the source codes or the demoextras, etc; \miguel{Never use 'etc' in the documentation. Explain better the reasons why the DR can stop an execution.} then, the DemoRunner inform the Core \miguel{it doesn't know about the Core} about the causes of the failure so the Core can take the best action in response \miguel{remove 'in response'.}. Another issue is dealing with timeout \miguel{you use the term before explaining it} that occurs when a demo exceeds a reasonable time for its execution, either because of problems in the code or simply because the execution time exceeds a reasonable time. This execution time can be indicated in the DDL of the demo offering the possibility that the editor decides what he \miguel{Do not use 'he' or 'she', but 'they'} considers reasonable time. Otherwise, the IPOL system will assign its own value and likewise, if the time set in the DDL is excessive or too short, the DemoRunner will modify this to a more reasonable value. \miguel{It's not clear what you write. Explain that the DR needs to control processes which might have gone out of control and might never finish. To avoid this, the DR monitors their execution and stops these malfunctioning processes when their execution time is over a security maximum.}
