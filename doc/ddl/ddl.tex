\section{The Demo Description Language}

current version is 1.0

\subsection{Introduction}
The Demo Description Language (DDL) is a textual description of a IPOL demo that 
allows to compile, generate a web interface, and run a demo. The description is 
written in JSON (JavaScript Object Notation) format, which is a standard format 
used in web applications. This language is evolving to allow a maximum number 
of demos to be described without the need to manually write HTML or Python code 
for a given demo. Each main key in the description file is described in the 
following sections.


{\bf Note:} in JSON format, always use double quotes around keys or string 
values, not single quotes.

%-------------------------------------------------------------------------------
\subsection{The \emph{general} section}
The general section describes global information about the demo. It is a set of 
(key,value) pairs, described in the following table. Many keys are derived from 
the static variables of the previous 'app' Python class.

\begin{longtable}{|>{\bf}L{\dimexpr 0.27\linewidth}|L{\dimexpr 
0.58\linewidth}|c|}
\hline
 \centering {key}     & \centering {\bf description} & {\bf req} 
\tabularnewline \hline \hline
 demo\_title         & demo title & yes\\ \hline
 input\_description  & description at the top of the input selection page 
                     & yes \\ \hline
 param\_description  & description at the top of the parameters page & yes
                      \\ \hline
 input\_nb           & number of inputs (currently only 1 is supported) & yes\\ 
\hline
 input\_max\_pixels & sets the maximal number of pixels of the input image, 
bigger images will be downsized, if 0 no resizing is done. The resizing as 
defined in the image class uses python PIL resizing with 'antialias' option. & 
yes  \\ \hline
 input\_dtype       & input image expected data type, used as parameter of 
image.convert() method. Possible values are '1x8i' and '3x8i' which are 
converted respectively to 'L' and 'RGB' for PIL.& yes  \\ \hline
 input\_max\_weight & max size (in bytes) of an input file, prevents uploading 
bigger files. & yes  \\ \hline
 input\_ext         & input image expected extention (ie. file format) & yes  
\\ \hline
 is\_test           & boolean (true/false), if true will be removed from demos 
if the server is in production mode.& yes  \\ \hline
 xlink\_article     & defines the link to the article webpage & yes  \\ \hline
\caption{Keys for the 'general' section ({\em req} means required).}
\end{longtable}

%-------------------------------------------------------------------------------
\subsection{The \emph{params} section}
The params section describes the set of parameters needed by a demo, their 
constraints and their visual appearance. It is defined as an array of sets, 
where each set contains (key,value) pairs.

\subsubsection{Common values}

\begin{longtable}{|>{\bf}L{\dimexpr 0.27\linewidth}|L{\dimexpr 
0.58\linewidth}|c|}
\hline
 \centering {key}     & \centering {\bf description} & {\bf req} 
\tabularnewline \hline \hline
 type  & parameter type, one of \{ label, range, selection\_collapsed \}.
                     & yes \\ \hline
 label & description of the parameter or contents if type is label. & yes
                      \\ \hline
\caption{Common keys for the 'params' section ({\em req} means required).}
\end{longtable}


\subsubsection{ \emph{label} type}
The label type does not need any other value. It can be used as a title to 
separate groups of parameters.

\subsubsection{ \emph{selection\_collapsed} type}


\begin{longtable}{|>{\bf}L{\dimexpr 0.27\linewidth}|L{\dimexpr 
0.58\linewidth}|c|}
\hline
 \centering {key}     & \centering {\bf description} & {\bf req} 
\tabularnewline \hline \hline
 id     & parameter name  & yes \\ \hline
 values & description of the parameter or contents if type is label. & yes
                      \\ \hline
 default\_value &  & yes \\ \hline
 value & & yes \\ \hline
\caption{Additional keys for the 'selection\_collapsed' type.}
\end{longtable}

\subsubsection{ \emph{range} type}

\begin{longtable}{|>{\bf}L{\dimexpr 0.27\linewidth}|L{\dimexpr 
0.58\linewidth}|c|}
\hline
 \centering {key}     & \centering {\bf description} & {\bf req} 
\tabularnewline \hline \hline
 id     & parameter name  & yes \\ \hline
 values & description of the parameter or contents if type is label. & yes
                      \\ \hline
 default\_value & & yes \\ \hline
 value & & yes \\ \hline
\caption{Additional keys for the 'range' type.}
\end{longtable}

%-------------------------------------------------------------------------------
\subsection{The \emph{results} section}

%-------------------------------------------------------------------------------
\subsubsection{ \emph{run\_again} type}

\subsubsection{ \emph{warning} type}
\subsubsection{ \emph{image} type}
\subsubsection{ \emph{repeat\_image} type}
\subsubsection{ \emph{gallery} type}
\subsubsection{ \emph{repeat\_gallery} type}
\subsubsection{ \emph{html\_text} type}
\subsubsection{ \emph{file\_download} type}

%-------------------------------------------------------------------------------
\subsection{The \emph{build} section}
\subsubsection{url}
The full url link to download the demo source code.
\subsubsection{srcdir}
Subdirectory from the extracted archive where the source code is located.
\subsubsection{binaries}
List of binaries and their associated paths relative to the source code path.
\subsubsection{flags}
Cmake compilation flags.
\subsubsection{scripts}
List of scripts and their associated paths relative to the source code path.

\subsection{The \emph{run} section}
\subsubsection{keywords}

\subsection{Examples}


\subsubsection{keywords}
