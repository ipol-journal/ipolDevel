\section{The Archive module}

\subsection{Abstract}
\label{sec:archive_introduction}
\paragraph{Introduction} \hspace{0pt} \\
The archive module is a standalone application destinated to communicate with other modules using webservices. It is designed to implement a stable, simple and scalable system for archiving all experiments done with IPOL.
\paragraph{Technologies used} \hspace{0pt} \\
The archive module is written in Python, is using the cherrypy framework for webservices, the mako template library for webpage rendering, the Python Image Library for thumbnails creations, and the python-magic library available on pip (not to be mistaken with python-magic5 which is the one available on default ubuntu's APT repositories).
\paragraph{Module composition} \hspace{0pt} \\
The module is composed of very few files, the code itself in module.py, a cherrypy configuration file ``archive.conf'', two mako HTML templates, and a database. It will also need 3 directories, respectively for storing blobs, thumbnails and logs. These directories are indicated in ``archive.conf'', if they are missing from it, the module will not start. If the directories indicated in the configuration file does not exists, they will be created at the launch of the module if the user has the necessary rights. Otherwise, the module will not start.
\subsection{Other section}
